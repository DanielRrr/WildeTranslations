\documentclass[a4paper]{article}
\usepackage{amsmath}
\usepackage[english,russian]{babel}
\usepackage[utf8]{inputenc}
\usepackage{amsthm}
\usepackage{amsfonts}
\usepackage{amssymb}
\usepackage{bussproofs}
\usepackage{mathtools}
\usepackage{verbatim}
\usepackage{dsfont}
\usepackage{mathabx}
\usepackage[all, 2cell]{xy}
\usepackage[all]{xy}
\usepackage{wasysym}
\usepackage{rotating}
\usepackage{geometry}
\usepackage{trfsigns}
\usepackage{cmll}
\usepackage{authblk}

\theoremstyle{defin}
\newtheorem{defin}{Definition}

\theoremstyle{theorem}
\newtheorem{theorem}{Theorem}

\theoremstyle{prop}
\newtheorem{prop}{Proposition}

\theoremstyle{lemma}
\newtheorem{lemma}{Lemma}

\theoremstyle{ex}
\newtheorem{ex}{Example}

\theoremstyle{col}
\newtheorem{col}{Corollary}
\usepackage{listings} 		% for source code
\date{}
\author{Оскар Уайлд}
\title{Соловей и роза}

\begin{document}
\maketitle

- Она сказала, что потанцует со мной, если я принесу ей красные розы, - вздохнул юный Студент, - но в моем саду нет ни одной алой розы.

Тем временем, в своем гнезде на ветках каменного дуба, его услышал Соловей, наблюдая из-за листьев с большим любопытством.

- Ни алой одной розы во всем моей саду, - рыдал студент, покрывая свои прекрасные глаза слезами. - Боже, счастье зависит от достаточно существенного числа мелочей! Я прочел все, что написали великие умы, и познал все тайны философии. Но какая-то красная роза попросту разрушила мою жизнь.

- Он все же по-настоящему влюблен, - сказал Соловей. - Каждую ночь я пел о нем, совсем его не зная. Каждую ночь я пел звездам
свои песни о нем, и теперь я наконец-то вижу его. Его черные волосы похожы на бутон гианцита, а губы так же красны, как и
вожделенная им роза. Он побледнел, что сделало его лицо похожим на слоновую кость, а печаль брови.

- Принц устраивает бал завтра вечером, - вполголоса говорил юный Студент, - и моя возлюбленная будет там. Если я подарю ей алую розу, то она будет танцевать со мной до рассвета. Если я подарю ей алую розу, я возьму ее за руки, она положит голову на мою грудь, а ее руки сожмутся в моих. Но, если же в моем саду не появится ни одной алой розы, то она пройдет мимо меня, оставив меня в одиночестве, а мое сердце будет разбито.

- Он действительно по-настоящему влюблен, - сказал Соловей, - он страдает от того, о чем я пел. Его боль --- это моя радость. Любовь по-настоящему прекрасна. Она намного дороже изумрудов, ее не купишь за жемчуга и рубины, она не продается на рынке.
Ее не обменяешь у ростовщиков, и ее не взвесишь, как золото.

- Музыканты будут играть в зале, - продолжил молодой Студент, - и будут играть на своих струнных инструментах, а моя возлюбленная будет танцевать под звуки арфы и скрипки. Она будет танцевать так легко, что едва будет доставать ногами до пола, и
все приглашенные гости в пышных нарядах окружат ее своими взорами. Но она не станет танцевать со мной, если я не принесу ей красную розу.

После чего он упал на газон и заплакал, закрыв руками свое лицо.

- Почему он плачет? - спросила маленькая зеленая Ящерица, пробежав за ним, вмахнув своим хвостом.

- Правда, почему? - вопрошала Бабочка, порхая своими крыльями над солнечными лучами?

- Правда, почему? - тихо спросила Ромашка своей соседке, очень мягко и тихо.

- Он плачет из-за красной розы, - ответил Соловей.

- Из-за красной розы? Что за глупость! - засмеялись все, а громче всех хохотала маленькая Ящерица, несколько склонная к определенной степени цинизма.

Но Соловей понял причину печали Студента, и, безмолвно сидя на ветвях дуба, рассуждал о тайне Любви.

Внезапно он взамухнул своими бурыми крыльями и воспарил в воздухе. Он, подобно тени, пролетел рощу и оказался в саду.

Посреди сада стоял прекрасный Розовый куст, и Соловей, как только увидев этот куст, порхнул к нему.

- Дай мне красную розу, - просил Соловей, - и я спою тебе свою самую прекрасную песню.

Но Куст лишь покачал головой.

- Мои розы белы, - ответил Куст, - Белы, как морская пена, и белее снега на вершине высокой гор. Иди к моему брату, что растет неподалеку от солнечных часов, возможно, он даст тебе то, что ты ищешь.

И Соловей полетел к Розовому кусту, что произрастал у старинных солнечных часов.

- Дай мне красную розу, - просил Соловей, - и я спою тебе свою самую прекрасную песню.

И этот Куст всего лишь покачал головой.

- У меня желтые розы, - сказал Куст, - Они желтые, как волосы русалки, что сидит на янтарном троне. Желтые, как нарцисс, что растет на лугах, пока крестьяне их не скашивают своими косами. Иди к моему брату, что растет под окнами Студента, и, быть может, он даст тебе, что ты хочешь.

И Соловей отправился к Кусту, что росли прямо под окнами Студента.

- Дай мне красную розу, - просил Соловей, - и я спою тебе свою самую прекрасную песню.

Куст покачал головой:

- Мои розы красны, - говорил Куст, - красны, как лапка голубя, они краснее, чем коралловый веер, что развевается в морской пещере. Но зима охладила мои вены, мороз потрепал мои бутоны, а мои ветки сломались после череды штормов. У меня больше не будет роз в этом году.

- Мне нужна всего лишь одна роза, - восклинул Соловей, - Всего лишь одна роза! Неужели нет способа достать ее?

- Есть один путь, - ответил Куст, - но он настолько ужасен, что я даже желаю тебе его сообщать.

- Скажи мне этот путь, - просил Соловей, - я не боюсь.

- Если тебе нужна роза, - начал Курс, - то ты должен создать ее своим пением в свете луны и окрасить ее своей собственной кровью. Ты должен петь, направив свою грудь напротив шипа. Всю ночь ты должен петь, и шипы пронзят твое сердце. Твоя кровь, попадя в мои вены, станет моей.

- Смерть --- это довольно дорогая плата за красную розу, - сказал Соловей, - и Жизнь не менее дорога. Так приятно сидеть в зеленом лесу и смотреть за Солнцем и золотой колесницей ее лучей, и за жемчужным покровом Луны. Как прекрасен запах бояршника, так чудесно благоухают спрятавшиеся в долине колольчики, и вереск, аромат которого ветром доносится с холмов. Даже Любовь прекраснее, чем Жизнь. И что такое сердце птицы по сравнению с жизнью человека?

Соловей снова впорхнул крыльями и воспарил в воздухе. Он, как тень, пронесся через сад, и, подобно же тени, пролетел через рощу.

Юный Студент все также лежал на траве, в том же месте, где Соловей оставил его, а его слезы так и не высохли под его прекрасными глазами.

- Будь счастлив, - говорил Соловей, - будь счастлив, ты получишь свою алую розу. Я создам ее своим пением под луной, и наполню ее жизнью своей собственной кровью из моего сердца. Все, что я прошу у тебя взамен, -- будь настоящим влюбленным. Любовь мудрее Философии, Любовь есть сама мудрость. Она могущественнее любой Силы, она есть само могущество. Ее крылья цвета пламени, и ее тело окрашено огнем. Ее губы сладкие, как мед, а дыхание подобно ладану.

Студент осмотрел луг вокруг себя и прислушался, но ничего не смог понять из того, кто Соловой говорил ему, он понимал только то, что прочел в своих книгах.

Но Дуб понял, что сказал Воробей, и загрустил, ведь Дуб очень любил Соловья, свившего на своих ветках гнездо.

- Спой мне последнюю песню, - шепнул Дуб, - Мне будет очень одиноко без меня.

И Соловей спел Дубу песню, и его голос звучал как вода, что льется из серебряного сосуда.

Когда он закончил петь свою песню, Студент встал и достал записную книжку и ручку из своего кармана.

- У него бесспорно есть форма, - говорил Студент самому себе, уходя из сада, - Но есть ли у нее чувство? Боюсь, что нет. По существу вопроса, он подобен большей части артистов, он сосредоточен вокруг стиля безо всякой искренности. Он не станет жертвовать собой ради других. Он думает просто о музыке, а искусство по своей природе эгоистично. Тем не менее, можно по-прежнему полагать, что в его голосе присутствуют неплохие моменты. Какая жалость, что они не понимают о чем поют и не приносят никакой практической пользы.

Он вошел в свою комнату, лег на свою маленькую кушетку и стал думать о любви. Спустя некоторое время, он заснул.

И когда Луна воцарилась на небосводе, Соловей полетел к Розовому Кусту, уставив свою грудь напротив шипа. Всю ночь он пел перед шипом, а холодная хрустальная Луна слушала его. Всю ночь он пел, а шип все сильнее пронизал его грудь, проливая все больше и больше крови.

Сначала он пел о том, как родилась любовь между мальчиком и девочкой, и на самой верхней ветке лепесток за лепестком распустилась чудесная роза, и так продолжалось песня за песней. Уже светало, и туман поднимался над рекой, серебряные крылья которого предзнаменовали рассвет. А, розы, лепестки которой распустились на самой верхней ветке, походила на собственную собственное отражение в серебряном зеркале.

Но Куст приказал Соловью пронзить свою грудь сильнее.

- Прислонись ближе! - велел Куст, - иначе день наступит до того, как роза будет готова.

И Соловей сильнее разрывал свою грудь. Он пел громче и громче о страсти, что возникает в сердецах юношей и девушек.

Листья розы слегка покраснели и стали похожи на румянец жениха, целующего губы своей невесты. Но шип еще не достал сердце Соловья, а сердце розы оставалось белым, но лишь кровь из Сердца Воробья могла обагрить бутон.

И Куст сказал еще Соловью прижать свою грудь к шипу.

- Прислонись ближе! - велел Куст, - иначе день наступит до того, как роза будет готова.

И Соловей прижался сильнее к шипу, и шип достал его сердце, и свирепая боль одолела Соловья. Боль становилась сильнее, как и его песня, что вещала о Любви под знаменем Смерти, о Любви, что живет после надгробья.

Очаровательная роза обагрилась, как восточный закат. Обагрились ее лепестки и сердце, ставшее походить на рубин.

Но голос Соловья ослабел, его маленькие крылья стали биться от боли, и темная пелена покрыла его взор. Угасала его песня, и Соловей почувствовал, как к горлу подступило удушье.

Тогда он стал петь свою последнюю трель. Бледная Луна, услышав Соловья, забыла про рассвет и застыла в небе. Алая роза, услышав Соловья, троекратно выросла и распустила свои лепестки в утреннем холоде. Эхо\footnote{Персонаж греческой мифологии} донесло песню Соловья до своей пурпурной пещеры и разбуло пастухов, что спали там. Трель донеслась до камышей, что росли вдоль берега реки, а они донесли послание морю.

- Смотрите, смотрите! - воскликнуло Дерево. - Роза завершилась в своем создании.

Но Соловей ничего не ответил, он лежал, замертво упавший, на травке с шипом в своем сердце.

Днем Студент открыл окно и оглянулся.

- Боже, это величайшая удача! - удивился Студент, наклонившись и сорвав розу. - Красная роза! Я никогда не видел столь более алой розы, чем эта! Она настолько прекрасна, и, я убежден, у нее есть длинное латинское название.

Тогда он надел шляпу и помчался к дому Профессора с розой в руке.

Дочь профессора сидела у дверного прохода и наматывала на катушку нитки голубого шелка, приютив у своих ног маленькую собачку.

- Ты говорила, что потанцуешь со мной, если я принесу тебе алая розу, - говорил Студент. - Эта роза самая красная из всех, что я видел. Ты сможешь сегодня вечером украсить себя ею, а наш совместный танец выразит всю мою любовь к тебе.

Но девушка лишь смутилось.

- К моему великому сожалению, эта роза не попойдет к моему наряду, - ответила она, - и, помимо всего прочего, сын местного управляющего подарил мне настоящие жемчуга, которые, как тебе известно, куда дороже каких-то там роз.

- Что ж, помяни мое слово, ты очень и очень неблагодарна, - озлобленно вспылил Студент.

Тогда он выбросил розу на водосток, где ее переехало колесо.

- Неблагодарная?! - рассердилась девушка. - Тогда я тебе скажу, что ты очень груб, и, к слову, кто ты вообще такой? Какой-то студент. У тебя даже серебряной пряжи на ботинках нет, когда как у сына управляющего она есть!

Тогда она встала из-за своего стула и удалилась домой.

- Как же глупа любовь, - говорил Студент, уходя домой. - От нее куда меньше пользы, чем от логики, поскольку Любовь ничего не способна доказать, а лишь говорит о вещах, которые даже не факт, что имеют место, завставляя верить в заведомо ложные явления. По факту, любовь не имеет никакого практического значения, а в наше время такое качество имеет роль, практичным должно быть все вокруг. Мне стоит вернуться к философии и заняться метафизикой.

Он вернулся в свою комнату, достал огромную пыльную книгу и принялся читать.


\end{document}
