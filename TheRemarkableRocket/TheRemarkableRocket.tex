\documentclass[a4paper]{article}
\usepackage{amsmath}
\usepackage[english,russian]{babel}
\usepackage[utf8]{inputenc}
\usepackage{amsthm}
\usepackage{amsfonts}
\usepackage{amssymb}
\usepackage{bussproofs}
\usepackage{mathtools}
\usepackage{verbatim}
\usepackage{dsfont}
\usepackage{mathabx}
\usepackage[all, 2cell]{xy}
\usepackage[all]{xy}
\usepackage{wasysym}
\usepackage{rotating}
\usepackage{geometry}
\usepackage{trfsigns}
\usepackage{cmll}
\usepackage{authblk}

\theoremstyle{col}
\newtheorem{col}{Corollary}
\usepackage{listings} 		% for source code
\date{}
\author{Оскар Уайлд}
\title{Замечательная ракета}

\begin{document}
\maketitle

Сын Короля собирался жениться, и всюду царила общая радость. Он ждал свою невесту целый год, и наконец она прибыла к нему.
Она была Русской Принцессой, и она провела весь путь в санях, запряженных шестью северными оленями. Сани походили на золотого лебедя, и между крыльями этого лебедя сидела хрупкая Принцесса. Она была окутана с ног до головы шубой из меха горностая, ее голова была покрыта серебристым платком, а ее лицо было столь же бледным, как и Снежный Дворец, где она всегда жила. Она была настолько бледной, что люди, видя ее проходящей по улице, постоянно изумлялись. "Она как белая роза!", восклицали они, бросая ей в ноги цветы со своих балконом.

У ворот Замка Принц ожидал ее. У него были мечтательные фиалковые глаза, а его волосы были словно из чистого золота. И, увидев принцессу, он встал на одно колено и поцеловал ей руку.

- Твой портрет был прекрасен, - шепнул Принц, - но ты намного прекраснее своего портрета.

После чего Принцесса покраснела от смущения.

- Раньше она была похоже на белую розу, - сказал юный Паж своему соседу, - а сейчас она будто алая роза.

И весь Двор пребывал в восторге.

Следующие три дня только об этом молва и ходила, "Белая роза, Красная роза, Красная роза, Белая роза", после чего Король издал указ об увеличении жаловья Пажу вдвое. Паж вовсе не получал никакого жалованья за ненадобностью, но это была такая честь, что о ней сразу же написали в Придворной газете.

Спустя три дня все кругом отмечали свадьбу. То была торжественная церемония, а невеста и жених шли, держась за руки, под  навесом из пурпурного бархата, украшенным крохотными жемчугами. Тогда началась Царская Трапеза, что продолжалась пять часов. Принц и Прицнесса сидели во главе стола в Главном зале и пили из бокалов, сделанных из чистого хрусталя. Только истинные возлюбленные могли пить из этих бокалов, а прикосновение губ лжеца может сделать эти бокалы серыми и тусклыми.

- Совершенно ясно, что они любят друг друга, - сказал юный Паж, - ясно, как хрусталь!

И Король снова удвоил его жаловонье.

- Какая часть! - воскликнули придворные.

После банкета все отправились на бал. Невеста и жених танцевали танец Розы, а Король захотел сыграть на флейте. Он очень плохо играл на флейте, но никто не осмеливался сказать ему это, потому что он был Король. Он знал всего лишь две сюиты, и никогда не мог точно сказать, какую именно из них он сейчас играет. Конечно же, это не играло никакой роли, и, что бы ни делал Король, все лишь изумлялись: "Очаровательно! Очаровательно!"

Следующим номером программы был грандиозный салют ровно в полночь. Принцесса никогда не видела прежде салюта в своей жизни, и Король приказал Придворному Пиротехнику быть на посту в день свадьбы.

- Как выглядят салюты? - спросила Принцесса Принца одним утром, гуляя по террасе.

- Они похожи на Аврора Бореалис \footnote{Полярное сияние}, - сказал Король, который обычно отвечал на вопросы, адресованные другим людям, - только они более естественные. Я сам предпочитаю салюты звездам, потому что вы всегда знаете, когда именно они появятся, и они столь же прелестны, как моя игра на флейте. Скоро вы все увидите сами. 


\end{document}
